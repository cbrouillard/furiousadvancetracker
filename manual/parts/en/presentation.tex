\subsection{Presentation and disclaimer}
\FAT is a musical software conceived for the Gameboy Adance.
With \FAT, you'll be able to make music directly on your console using its technical capacities.
\FAT is trying to be the simplest as possible but you'll probably have to learn how to use it in order to exploit all the features it provides.
\medskip

The current version, \fatversion, is the first really stable.
I tried to make it the safest and bugfree as possible but this software is probably a little bit buggy or not finalized.
I'm currently working on \FAT (well ... while I get motivation, I'll continue to develop it) so if you found a bug or an unusual case, please report me the issue.
It's really important to make the software continue to grow. You can report any issue on \url{http://wwww.furiousadvancetracker.com}.
Many thanks to you ! (and a kiss. But not a french one :D)
\medskip

The \FAT's GUI really looks like LSDJ. Actually, I considered LSDJ as my model when designing \FAT while adding my personals touch and ideas.
Some features are completely originals when others has disapeared.
\medskip

Of course I'm not pretending \FAT is as performant as LSDJ. I'm working hard to make this possible one day and I hope \FAT will be an important software for the GBA chiptune's community.
\medskip

So ... a great thanks to you, who are reading this manual. Thanks for trying \FAT, hope you'll love it !

\subsection{Why \FAT ?}
When I started this project, in 2011, severals solutions already exist allowing you to compose music stuff on your GBA.
\medskip

\begin{itemize}
    \item{Nanoloop 2.3/2.5}
    \item{M4G tracker}
    \item{and problably many others...}
\end{itemize}\medskip

In my opinion, all of these seems to be incomplete although all of them are reasonnably usable.
\medskip

\begin{itemize}
    \item{Nanoloop 2.3/2.5 is an excellent software ! but it's quite difficult to get your hands on. It's also quite expensive (due to rarity probably).}
    \item{M4G tracker is also a good software: but it's development seems to be stopped. Well, like \FAT a little while ago.}
    \item{I didn't find any other solutions...}
\end{itemize}\medskip

Finally you know what ? Coding my own tracker has proven to be really fun !
\medskip

\subsection{How to use \FAT ?}

\FAT is a gba file. This file is directly usable on your computer or smarphone with a GameboyAdvance emulator (such as VisualBoyAdvance).
\medskip

Anyway, an emulated sound can never be equal than the true one.
To execute \FAT on a real GBA, you'll need :
\medskip

\begin{itemize}
    \item{a true GBA or a Gameboy Micro. I personaly prefer the second one but some says the screen is too small.}
    \item{an EZ-Flash, SuperCardToSD, or other cartridge ... You'll have to search the web in order to find one}
    \item{a SD card in the case your cartridge use one.}
\end{itemize}\medskip

When you have bought all this cool stuff, all you have to do is to paste the \FAT gba file {\bf (with the provided cartridge's software)}.
It's really important to follow all the cartridge's constructor's instructions : \FAT could not work correctly if you don't.

\ColoredAnnotation{Warning ! Although this version has been considered as stable, it remains experimental. I thinks you don't have to waste money in order to try \FAT but it's your choice !}

If you really want to buy you own cartridge, the following sheet can help you.

\subsection{GameboyAdvance cartridge comparison sheet}

There's a lot of GBA cartridges. These cartridges allow you to execute "homebrew" program like \FAT.
Here are some cartridges tested with \FAT.

% Configuration du tableau
\tablehead{\hline \rowcolor{headertab} {\bf Cartridge} & {\bf Burning} & {\bf Starting} & {\bf Saving} \\ \hline}
\tabletail{\hline \multicolumn{4}{|l|} {\small ...} \\ \hline}
\tablelasttail{\hline}
\begin{supertabular}{|p{3.5cm}|l|l|l|}
    \hline
        \begin{minipage}[c]{3cm}
        \vspace{0.5cm}
        \SimpleImage{images/supercardsd_djangofeet_sd}{0.2}
        \end{minipage} &
        \begin{minipage}{3cm}
        Use the official software
        \end{minipage} &
        \begin{minipage}{2cm}
        \textcolor{vert}{OK}
        \end{minipage} &
        \begin{minipage}{7cm}
        The save features is working but you'll have to press R+L+A+START after each save.
        \end{minipage} \\
    \hline
        \begin{minipage}[c]{3cm}
        \vspace{0.5cm}
        \SimpleImage{images/supercardminisd_djangofeet_sd}{0.2}
        \end{minipage} &
        \begin{minipage}{3cm}
        Use the official software
        \end{minipage} &
        \begin{minipage}{2cm}
        \textcolor{bleu}{NOT TESTED}
        \end{minipage} &
        \begin{minipage}{7cm}
        Not tested but should be similar with the SuperCardSD above.
        \end{minipage} \\
    \hline
        \begin{minipage}[c]{3cm}
        \vspace{0.5cm}
        \SimpleImage{images/ezflash1}{0.2}
        \end{minipage} &
        \begin{minipage}{3cm}
        Use the official software and his USB linker
        \end{minipage} &
        \begin{minipage}{2cm}
        \textcolor{vert}{OK}
        \end{minipage} &
        \begin{minipage}{7cm}
        Nothing to report, all is working ! This is far the better choice.
        \end{minipage} \\
    \hline
        \begin{minipage}[c]{3cm}
        \vspace{0.5cm}
        \SimpleImage{images/ultraflash2advance}{0.2}
        \end{minipage} &
        \begin{minipage}{3cm}
        Use the official software, GBA will be your linker
        \end{minipage} &
        \begin{minipage}{2cm}
        \textcolor{vert}{OK}
        \end{minipage} &
        \begin{minipage}{7cm}
        Save feature is working.
        \end{minipage} \\
    \hline
        \begin{minipage}[c]{3cm}
        \vspace{0.5cm}
        \SimpleImage{images/supercardsd_sd}{0.2}
        \end{minipage} &
        \begin{minipage}{3cm}
        ???
        \end{minipage} &
        \begin{minipage}{2cm}
        \textcolor{rouge}{KO}
        \end{minipage} &
        \begin{minipage}{7cm}
        Not tested. Some users already reported this cartridge is not working with \FAT :(
        \end{minipage} \\
    \hline
        \begin{minipage}[c]{3cm}
        \vspace{0.5cm}
        \SimpleImage{images/ezflash4}{0.2}
        \end{minipage} &
        \begin{minipage}{3cm}
        \FAT is starting and is usable. Use the provided software to burn the ROM.
        \end{minipage} &
        \begin{minipage}{2cm}
        \textcolor{vert}{OK}
        \end{minipage} &
        \begin{minipage}{7cm}
        Save doesn't work ! Please use the instant save provided by the cartridge.
        \end{minipage} \\
\hline
\end{supertabular}

\medskip The better choice is the ezflashI.

\subsection{Changelog for \fatversion}

Changelog for this \FAT's version: \medskip
\begin{itemize}
  \item{general stability improvements (all the code has been reworked)}
  \item{the samples and oscillator playback had been recoded (now it's stable)}
  \item{added a new screen : custom wave editor}
  \item{added a new project parameter : the sample rate determines the playback speed for samples and oscillators}
\end{itemize}

\subsection{Next version roadmap}

The next version should be \fatnextversion. \medskip
\begin{itemize}
    \item{new commands}
    \item{total rework of custom wave screen}
    \item{seeking some help/documentation for synth FM}
\end{itemize}

\subsection{First sound}

Let's start easy ! We gonna play our first sound with \FAT.\medskip

When \FAT starts, the SONG screen appears first. Don't move the curser, simply press on 'A' button.
A "00" magically appears: this is a {\bf sequence}.

\Image{images/screen_song}{1.0}{SONG screen}

Press the SELECT button (hold it) et move to the right only once : the interface has changed.

\Image{images/screen_popup}{1.0}{POPUP screen}

Now, without to move the cursor, press again on the 'A' button. A new "00" appears but this time, it's a {\bf block}.

\Image{images/screen_blocks}{1.0}{BLOCKS screen}

Press SELECT (hold it) and move to the right. Welcome to a new screen !
Press 'A' another time : here is your first {\bf note} !
Let's hear it : simply press START. \FAT is currently playing your first track on a GBA. Enjoy.

\Image{images/screen_notes}{1.0}{NOTES screen - press START !}

\subsection{Data structure}

Ok, we have the "sequences", "blocks" and "notes" notions in mind. \FAT is globally organized the same as LSDJ :
\begin{itemize}
    \item{a sequences table on 6 columns : GBA can natively play 6 sounds at the same time.}
    \item{for each sequence, we have 16 blocks.}
    \item{for each block, we have 16 notes.}
\end{itemize}\medskip

The way to organize all the data is like "russians dolls". Sequences contains blocks which themselves contains notes.
When \FAT is playing a sequence, it's running all blocks in it one after others.

In order to avoid each notes ring like others, we "plug" an "instrument" on it.
Each instrument provides you some parameters.
If you change instrument's parameters, then your notes will ring differently.
You also have to know that \FAT provides you several types of instruments ! allowing you to make differents things on the GBA's channels.

\subsection{Technical data}

The GameboyAdvance provide us 4 "analogic" channels in order to generate sounds and noises.
These channels are almost the same as on the first Gameboy (the greybrick).
In addition, GBA has 2 "numeric" channels : use them to play some elaborate sounds.

GBA's speaker is mono ! You'll have to plug your earphones to enjoy \FAT's stereo capacities.

\paragraph{Channel 1 : PU1} This channel allow you to play melody and/or apply SWEEP effect on notes. Beware that the channel 1 is the only channel able to make SWEEP.

\ColoredAnnotation{For example, you can use this channel to generate a kick with this SWEEP param or play bassy bass.}

\paragraph{Channel 2 : PU2} This is exactly the same as the PU1 but without the SWEEP capability.

\ColoredAnnotation{You can use this channel to play another bass or a melody.}

\paragraph{Channel 3 : WAVE} Channel 3 is able to play wave sound. It offers some cool capacities.

\ColoredAnnotation{Make this channel a way to get more bass on your song or play some catchy satured sound, like a cool electric guitar or so.}

\paragraph{Channel 4 : NOISE} Channel 4's purpose is to make some noise.

\ColoredAnnotation{Use it to make some snare. You can also program some cool special effects such as beach wave sound for example.}

\paragraph{Channel 5 : SOUNDA} Play samples and loops with this channel.

\paragraph{Channel 6 : SOUNDB} Play samples and loops with this channel.

\ColoredAnnotation{These two channels are perfect to add some human voices or better kick and snare. It's fully possible to add your own samples and use it within \FAT.}
