\ColoredAnnotation{\FAT n'est pas encore très riche d'effets en tout genre pour le moment: il va falloir s'en contenter pour l'heure.
Plusieurs commandes sont aux programme et débarqueront dans les versions suivantes. Patience !}

Voici les commandes actuellements utilisables.

\subsection{HO (Hop!)}

Cette commande permet de sauter directement au block suivant.
Elle peut permettre de créer des décalages de temps entre les blocks.
La valeur que vous paramétrez correspond au numéro de ligne vers lequel sauter.

\subsection{KL (KilL)}

Cette commande permet de stopper (kill = tuer) le son.

\ColoredAnnotation{Pas de valeur possible pour le moment pour ce paramètre.}

\subsection{OU (OUtput)}

Dirige le son vers la droite (\_R), gauche (L\_), les deux (LR) ou aucun des deux (\_\_)

\subsection{SW (SWeep)}

Cette commande n'est effective que sur le channel 1 (PU1) : elle permet de modifier la valeur du sweep.
Ce sweep remplace celui configuré dans l'instrument.

\subsection{VO (VOlume)}

Cette commande permet de modifier le volume du son.
\#captainobvious.
Notez que selon le channel, la valeur que vous passez est prise en compte différemment.
