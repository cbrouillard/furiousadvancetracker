Dans cet écran, vous pouvez éditer le contenu d'une séquence : chaque séquence peut contenir de 0 à 16 blocks.
Lorsque \FAT jouera la séquence, elle commencera par lire le premier block pour descendre jusqu'au dernier emplacement vide.
Si \FAT rencontre un emplacement vide lors de la lecture alors il remonte au premier emplacement.
Si aucun block n'est présent dans la séquence en cours, alors \FAT ne la jouera pas (elle sera sautée).
\medskip

Pour faire simple, considérez cet écran comme une liste de blocks contenue dans une séquence (dont le numéro est affiché à droite de l'écran).

\Image{images/screen_blocks_complete}{1.0}{BLOCKS screen}

\subsubsection{Commands}
% Configuration du tableau
\tablehead{\hline \rowcolor{headertab} {\bf Key(s)} & {\bf Effect} \\ \hline}
\tabletail{\hline \multicolumn{2}{|l|} {\small ...} \\ \hline}
\tablelasttail{\hline}
\begin{supertabular}{|l|p{11cm}|}
\hline
    {\bf SELECT} & Show the switch popup \\
    \hline
    {\bf R+L} & Show the help screen \\
    \hline
    {\bf START} & Start/stop blocks playback from currently edited sequence \\
    \hline
    {\bf L+HAUT} & Move the cursor to top \\
    \hline
    {\bf L+BAS} & Move the cursor to bottom \\
    \hline
    {\bf A} & Write last known block's number on the spot designated by cursor (default is "00") \\
    \hline
    {\bf A+DIRECTION} & Modify value \\
    \hline
    {\bf L+A} & Find a new available number and set the value on spot \\
    \hline
    %{\bf R+A} & Non utilisé pour le moment \\
    %\hline
    {\bf B} & IF NOT EMPTY SPOT | Cut block \\
    \hline
    {\bf B} & IF EMPTY SPOT | Paste block \\
    \hline
    {\bf L+B} & IF NOT EMPTY SPOT | Change number of block while copying its content \\
    \hline
    {\bf L+B} & IF EMPTY SPOT | Paste block but changing number \\
    \hline
    {\bf R+DROITE/GAUCHE} & Change the currently edited sequence number \\
    %\hline
    % {\bf R+HAUT/BAS} & Sauter à la séquence précédente/suivante dans le séquenceur \\
\hline
\end{supertabular}
