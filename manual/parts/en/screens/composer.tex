Imagine this screen as a mini-piano or a sampler machine : COMPOSER screen is going to transform your Gameboy Advance into a sound manager machine !

\Image{images/screen_composer}{1.0}{COMPOSER screen}

Pinciple is quite simple.
The screen shows you 8 note's spots.
Each note has been affected to a physical button (only the SELECT and START buttons are not usable).
Look at to bottom of screen, a word "UNLOCKED" appears :
    it means you are in "edition" mode. You can write any notes like you would do in screen BLOCK (with A button).
\medskip

Once you have written all your notes (each notes can have different instrument), press START button.
Mode's value change to "LOCKED" : you are now in "player" mode.
\medskip

Appuyez sur une touche: la note assignée est jouée en temps réel.
Changez de touche pour jouer une autre note. Simple n'est ce pas ?
\medskip

Notez aussi que ces données seront enregistrées dans chacune de vos chansons.
Vous pourrez préparer votre set de note à la maison et improviser en live sans forcer !
\medskip

Essayez ! composez quelques séquences puis essayez de jouer avec le COMPOSER en même temps !
\medskip

\subsubsection{Paramétrage}

Le composer peut également être paramétré:

\Image{images/parameters_composer}{1.0}{Les paramètres de l'écran COMPOSER}

\paragraph{Transpose} de 0 à FF.
Ajoute une valeur de transpose.
Cette valeur s'ajoutera à celle configurée pour le projet dans sa globalité.
Notez que toutes les notes du composer seront transposées avec cette valeur.

\paragraph{Key repeat} de 0 à FF.
Règle le temps d'attente entre chaque lecture d'une note du composer.
Plus cette valeur est élevée, plus le temps d'attente entre l'appui de deux touches sera grand.
Laissez cette valeur à zéro pour désactiver le temps d'attente.

\subsubsection{Commandes}

% Configuration du tableau
\tablehead{\hline \rowcolor{headertab} {\bf Touche(s)} & {\bf Effet} \\ \hline}
\tabletail{\hline \multicolumn{2}{|l|} {\small ...} \\ \hline}
\tablelasttail{\hline}
\begin{supertabular}{|l|p{11cm}|}
\hline
    {\bf SELECT} & Passer en mode popup \\
    \hline
    {\bf R+L} & Afficher l'écran d'aide \\
    \hline
    {\bf START} & Changer le mode (UNLOCKED/LOCKED) \\
    \hline
    {\bf TOUCHES} & LOCKED | Jouer la note correspondante \\
    \hline
    {\bf A+DIRECTION} & UNLOCKED | Changer la valeur de la note/instrument \\
\hline
\end{supertabular}
