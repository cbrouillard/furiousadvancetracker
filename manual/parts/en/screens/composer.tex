Imagine this screen as a mini-piano or a sampler machine : COMPOSER screen is going to transform your Gameboy Advance into a sound manager machine !

\Image{images/screen_composer}{1.0}{COMPOSER screen}

Pinciple is quite simple.
The screen shows you 8 note's spots.
Each note has been affected to a physical button (only the SELECT and START buttons are not usable).
Look at to bottom of screen, a word "UNLOCKED" appears :
    it means you are in "edition" mode. You can write any notes like you would do in screen BLOCK (with A button).
\medskip

Once you have written all your notes (each notes can have different instrument), press START button.
Mode's value change to "LOCKED" : you are now in "player" mode.
\medskip

Press any key : \FAT play assigned note in realtime.
Press another key to play another note ! Easy isn't it ?
\medskip

Take note that all data stored in the composer will be saved in your tracks.
That's why you can setup your note's collection at home and then rocks your improvisation on the stage without effort !
\medskip

Try it ! Compose some cool stuff then try to play with the composer while \FAT is playing your track.
\medskip

\subsubsection{Commands}

% Configuration du tableau
\tablehead{\hline \rowcolor{headertab} {\bf Key(s)} & {\bf Effect} \\ \hline}
\tabletail{\hline \multicolumn{2}{|l|} {\small ...} \\ \hline}
\tablelasttail{\hline}
\begin{supertabular}{|l|p{11cm}|}
    {\bf SELECT} & Show the switch popup \\
    \hline
    {\bf R+L} & Show the help screen \\
    \hline
    {\bf START} & Change composer mode (UNLOCKED/LOCKED) \\
    \hline
    {\bf KEYS} & LOCKED | Play assigned note \\
    \hline
    {\bf A+DIRECTION} & UNLOCKED | Change value \\
\end{supertabular}

\subsubsection{Parameters}

There are some parameters you can set on the composer screen :

\Image{images/parameters_composer}{1.0}{COMPOSER screen's parameters}

\paragraph{Transpose} from 0 to FF.
Add a transpose value. This value will be added to all other values configured in the project.
All composer's notes is going to be transposed with this value.

\paragraph{Key repeat} from 0 to FF.
Setup the key repeat : a time value waited between each note when you hold buttons.
A high value means a bigger wait between two keys. Leave with zero to disable this feature.
