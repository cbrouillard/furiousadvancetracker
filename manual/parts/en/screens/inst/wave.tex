\Image{images/screen_instrument_wave}{0.5}{WAVE type}

\subsubsection{Parameters}

\paragraph{Enveloppe : Volume} from 0 to 4.
This value adjusts the volume of notes attached to this instrument.

\paragraph{Parameter : "Timed"} 0 or 1.
\medskip

\begin{itemize}
    \item{0: the sound is not timed}
    \item{1: the sound is timed, set the duration with the parameter "Soundlength"}
\end{itemize}

\paragraph{Parameter : "Soundlength"} from 0 to FF.
This parameter can only be accessed if "Timed" has been set to 1.
\medskip

\begin{itemize}
    \item{00: sound is "infinite"}
    \item{EB: sound is short}
    \item{FF: sound is really, really short}
\end{itemize}

\paragraph{Parameter : "Voice"} from 0 to 17.
\FAT includes 23 types of WAV instruments (hexadecimal, 17h = 23).
Since version 1.2.0, it is possible to customize and use its own voices.

\paragraph{Parameter : "Bank"} 0 or 1.
The GBA's WAV channel is capable of loading 2 banks into its memory.
You can decide which bank will be played,
Knowing that the bank is directly loaded from the "Voice" parameter just above (each "Voice" has 2 banks).

\paragraph{Parameter : "Bankmode"} "SIN" or "DUA".
The WAV channel is able to synthesize the 2 banks loaded in memory (DUA mode = Dual) or to play only one (the SIN = Single mode).
Note that if you set this parameter with SIN, \FAT will play the selected bank with the "Bank" parameter.

\paragraph{Parameter : "Output"} Select the sound output mode.
\medskip

\begin{itemize}
    \item{L: the sound goes out to the left (LEFT)}
    \item{R: the sound goes out to the right (RIGHT)}
    \item{RL: the sound goes out to the left and right (LEFT/RIGHT)}
    \item{VIDE: sound does no go out anymore}
\end{itemize}

\paragraph{Custom wave} \FAT allows you to store and use 3 customized voices.
For more information, see \hyperref [subsec: customwave] {"CUSTOM WAVE screen"}.
\medskip

\begin{itemize}
  \item{an 'X' means that the instrument does not use a customized voice. The voice defined by \FAT is therefore used.}
  \item{change this 'X' to a value between 0 and 2 to use the corresponding voice}
  \item{this selected voice can be edited by pressing "Go" (just below). \FAT will display the edit screen for the selected voice.}
  \item{if a voice has never been edited before, it is initialized with the current voice of the instrument.}
\end{itemize}\medskip

\tablehead{\hline \rowcolor{headertab} {\bf Key(s)} & {\bf Effect} \\ \hline}
\tabletail{\hline \multicolumn{2}{|l|} {\small ...} \\ \hline}
\tablelasttail{\hline}
\begin{supertabular}{|l|p{11cm}|}
    \hline
    {\bf A+LEFT/RIGHT} & Change custom wave number \\
    \hline
    {\bf A+UP/DOWN} & Cancel customization on the instrument (reset to 'X') \\
\hline
\end{supertabular}
\medskip

\paragraph{Simulator} Change the note written in this space:
         it will be played with the parameters of the instrument you are editing.
