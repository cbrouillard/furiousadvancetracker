Here you can play your song as you like.
It is this mode that is to be privileged if you want to wet the shirt on stage.
If you prefer to simply press START in the SONG screen and dance with the audience, free to you!
Note however that the LIVE mode is there to allow you to modify the mechanics of your songs.

\Image{images/screen_live}{1.0}{The LIVE screen appears like the SONG screen}

The sequencer is displayed again but this time there is no question of editing the sequences on the board.
However, you will be able to start each channel in the order that suits you.
Once a channel is started, it is possible to perform several actions :
\medskip

\begin{itemize}
    \item{change channel volume}
    \item{change channel transpose}
    \item{stop channel playback}
\end{itemize}
\medskip

Note that the LIVE screen has two modes: the MANUAL (MAN) mode and the AUTO mode.
In MANUAL mode, FAT does NOT progress in the sequencer. The same sequence on the channel is replayed infinitly: it is up to you to advance the piece.
Conversely, in AUTO mode, FAT behaves as in the SONG screen: all channels currently playing progress normally.
\newpage % hack pourri
\subsubsection{Commands}
\begin{supertabular}{|l|p{10cm}|}
    {\bf SELECT} & Show the switch popup \\
    \hline
    {\bf R+L} & Show the help screen \\
    \hline
    {\bf START} & Lance/arrête la lecture du channel sous le curseur \\
    \hline
    {\bf R+START} & Lance/arrête la lecture de tous les channels \\
    \hline
    {\bf R+BAS} & Passer dans la partie configuration de l'écran (modification du volume, tsp, tempo et mode) \\
    \hline
    {\bf R+HAUT} & Passer dans la partie séquenceur de l'écran \\
    \hline
    {\bf A+DIRECTION} & DANS LA PARTIE CONFIGURATION - Modifie la valeur du paramètre sous le curseur \\
    \hline
    {\bf A+L+DIRECTION} & DANS LA PARTIE CONFIGURATION - Modifie la valeur du paramètre pour tous les channels \\
\end{supertabular}

\subsubsection{Précisions sur l'application du volume}

Lorsque vous changez le volume sur le channel, \FAT calcule un nouveau volume à appliquer sur chaque note.
\medskip

\begin{itemize}
    \item{si la valeur est "df" dans l'écran LIVE, alors le volume de l'instrument ou celui d'une éventuelle commande est appliqué. "df" = "defined"}
    \item{si la valeur est autre que "df", alors le volume est une moyenne calculée entre la valeur écrite dans l'instrument (ou dans une commande) et celle demandée dans le LIVE}
\end{itemize}
\medskip

Un volume de 00 n'éteint donc pas forcément le channel ! Il le divise par 2.
Il n'y a pour le moment pas de mode MUTE.

\ColoredAnnotation{Cette méthode d'application du volume sera peut-être ammené à changer. Si vous avez un avis, faites le moi savoir.}

\subsubsection{Précisions sur l'application du transpose}

Comme pour le volume, \FAT calcul la valeur de transpose à appliquer en fonction d'autres paramètres.
La valeur de transpose à prendre en considération se calcule très simplement comme suit :

\ColoredAnnotation{transpose du projet + transpose sur le block + transpose sur le live = transpose}
