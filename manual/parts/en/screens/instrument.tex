This screen is subdivided into 5 parts: there are as many types of instruments in \FAT.
\medskip

\begin{itemize}
    \item{PULSE}
    \item{WAVE}
    \item{NOISE}
    \item{KIT/SAMPLE}
    \item{OSCILLATOR}
\end{itemize}\medskip

Each instrument is specialized in playing notes {\bf on a certain channel}.
Remember, the SONG screen is composed of 6 columns.
These 6 columns actually represent each of the 6 channels of the GBA. So,
 \medskip

 \begin{itemize}
    \item{PULSE instruments are specialists in channel 1 and 2: the PU1 and PU2}
    \item{the WAVE instruments are useful for channel 3 only: the WAV channel}
    \item{the NOISE instrument is intended for channel 4: the NOI channel}
    \item{the instruments of type KIT / SAMPLE officiate on channel 5 and 6: the SNA and SNB}
    \item{finally, the OSCILLATOR softly generate a basic sound (sinusoidal, square, triangular or sawtooth)}
 \end{itemize}\medskip

 But \FAT is flexible enough to allow the use of an instrument in a channel that is not dedicated to it:
   The behavior can be surprising ... or inaudible.
   It's up to you to see what you want to produce as sounds.

\subsubsection{Commandes}

% Configuration du tableau
\tablehead{\hline \rowcolor{headertab} {\bf Touche(s)} & {\bf Effet} \\ \hline}
\tabletail{\hline \multicolumn{2}{|l|} {\small ...} \\ \hline}
\tablelasttail{\hline}
\begin{supertabular}{|l|p{11cm}|}
\hline
    {\bf SELECT} & Show the switch popup \\
    \hline
    {\bf R+L} & Show the help screen \\
    \hline
    {\bf START} & Start/stop blocks playback from currently edited block \\
    \hline
    {\bf B} & Play note in simulator \\
    \hline
    {\bf B+DIRECTION} & Change simulator's value \\
    \hline
    {\bf A+DIRECTION} & Modify value \\
    \hline
    {\bf R+DROITE/GAUCHE} & Change currently edited instrument's number \\
    \hline
    {\bf L+DROITE/GAUCHE} & Change instrument's type \\
\hline
\end{supertabular}

\subsubsection{Instrument de type PULSE}
\Image{images/screen_instrument_pulse}{0.5}{Type PULSE}

\subsubsection{Paramètres}

\paragraph{Enveloppe: "Volume"} de 0 à F.
Cette valeur permet de régler le volume des notes attachées à cet instrument.
\medskip

\begin{itemize}
    \item{0: muet}
    \item{F: à fond !}
\end{itemize}

\paragraph{Enveloppe: "Direction"} \SimpleImage{images/envelope_direction_down}{1.0} ou \SimpleImage{images/envelope_direction_up}{1.0}.
Indique si le son monte ou descend.

\paragraph{Enveloppe : "Steptime"} de 0 à 7.
Agit sur la longueur du pas de l'onde (en gros).
\medskip

\begin{itemize}
    \item{0: paramètre non pris en compte}
    \item{1: assez court}
    \item{7: très court}
\end{itemize}

\paragraph{Enveloppe : "Wave"} \SimpleImage{images/envelope_waveduty_0}{1.0} ou \SimpleImage{images/envelope_waveduty_1}{1.0} ou \SimpleImage{images/envelope_waveduty_2}{1.0} ou \SimpleImage{images/envelope_waveduty_3}{1.0}.
Modifie la forme de l'onde jouée par la GameboyAdvance.

\paragraph{Paramètre : "Timed"} 0 ou 1.
\medskip

\begin{itemize}
    \item{0: le son n'est pas temporisé}
    \item{1: le son est temporisé, réglez la durée avec le paramètre "Soundlength"}
\end{itemize}

\paragraph{Paramètre : "Soundlength"} de 0 à 3F.
Ce paramètre n'est accessible que si "Timed" à été positionné sur 1.
\medskip

\begin{itemize}
    \item{0: le son est "infini"}
    \item{3f: le son est très (très... voire trop) court}
\end{itemize}

\paragraph{Paramètre : "Output"} Choisissez le mode de sortie du son.
\medskip

\begin{itemize}
    \item{L: le son sort à gauche (LEFT)}
    \item{R: le son sort à droite (RIGHT)}
    \item{RL: le son sort à droite et à gauche (LEFT/RIGHT)}
    \item{VIDE: le son ne sort plus (muet)}
\end{itemize}

\paragraph{Paramètre : "Sweep"} de 0 à 7F.
Ajoute un effet particulier sur la note (le mieux est encore de tester).
N'oubliez pas que ce paramètre ne s'applique sur sur le channel PU1.

\paragraph{Simulator} Changez la note inscrite dans cet espace:
        celle-ci sera jouée avec les paramètres de l'instrument que vous êtes en train d'éditer.


\subsubsection{Instrument de type WAVE}
\Image{images/screen_instrument_wave}{0.5}{Type WAVE}

\subsubsection{Paramètres}

\paragraph{Enveloppe : Volume} de 0 à 4.
Cette valeur permet de régler le volume des notes attachées à cet instrument.

\paragraph{Paramètre : "Timed"} 0 ou 1.
\medskip

\begin{itemize}
        \item{0: le son n'est pas temporisé}
        \item{1: le son est temporisé, réglez la durée avec le paramètre "Soundlength"}
    \end{itemize}

\paragraph{Paramètre : "Length"} de 0 à FF.
Ce paramètre n'est accessible que si "Timed" a été positionné sur 1.
\medskip

\begin{itemize}
        \item{00: le son est "infini"}
        \item{eb: le son commence à être court}
        \item{ff: le son est très (très très) court}
    \end{itemize}

\paragraph{Paramètre : "Voice"} de 0 à 17.
\FAT inclut 23 types d'instruments WAV (en héxadécimal, 17h = 23).
Depuis la version 1.2.0, il est possible de customiser et d'utiliser ses propres voix.

\paragraph{Paramètre : "Bank"} 0 ou 1.
Le canal WAV de la GBA est capable de charger 2 banks dans sa mémoire.
Vous pouvez décider quelle bank sera jouée,
sachant que la bank est directement chargée depuis le paramètre "Voice" juste au dessus (chaque "Voice" possède 2 banks).

\paragraph{Paramètre : "Bankmode"} "SIN" ou "DUA".
Le canal WAV est capable de faire une synthèse des 2 banks chargées en mémoire (le mode DUA = Dual) ou bien de n'en jouer qu'une seule (le mode SIN = Single).
Notez que si vous réglez ce paramètre avec SIN, \FAT jouera la bank sélectionnée avec le paramètre "Bank".

\paragraph{Paramètre : "Output"} Choisissez le mode de sortie du son.
\medskip

\begin{itemize}
    \item{L: le son sort à gauche (LEFT)}
    \item{R: le son sort à droite (RIGHT)}
    \item{RL: le son sort à droite et à gauche (LEFT/RIGHT)}
    \item{VIDE: le son ne sort plus (muet)}
\end{itemize}

\paragraph{Custom wave} \FAT vous permet de stocker et d'utiliser 3 voix customisées.
Pour plus d'informations, référez vous à la section \hyperref[subsec:customwave]{"L'écran CUSTOM WAVE"}.
\medskip

\begin{itemize}
  \item{un 'X' signifie que l'instrument n'utilise pas de voix customisée. La voix définit par \FAT est donc utilisée.}
  \item{changez ce 'X' pour une valeur comprise entre 0 et 2 pour utiliser la voix correspondante}
  \item{cette voix sélectionnée peut être éditée en appuyant sur "Go" (juste en dessous). \FAT vous affiche alors l'écran d'édition pour la voix sélectionnée.}
  \item{si une voix n'a jamais été éditée auparavant, elle est initialisée avec les données de la voix courante de l'instrument.}
\end{itemize}\medskip

\tablehead{\hline \rowcolor{headertab} {\bf Touche(s)} & {\bf Effet} \\ \hline}
\tabletail{\hline \multicolumn{2}{|l|} {\small ...} \\ \hline}
\tablelasttail{\hline}
\begin{supertabular}{|l|p{11cm}|}
    \hline
    {\bf A+DROITE/GAUCHE} & Changer le numéro de voix custom sélectionnée \\
    \hline
    {\bf A+HAUT/BAS} & Annuler la customisation sur l'instrument (reset à 'X') \\
\hline
\end{supertabular}
\medskip

\paragraph{Simulator} Changez la note inscrite dans cet espace:
    celle-ci sera jouée avec les paramètres de l'instrument que vous êtes en train d'éditer.


\subsubsection{Instrument de type NOISE}
\Image{images/screen_instrument_noise}{0.5}{Type NOISE}

\subsubsection{Paramètres}

\paragraph{Enveloppe : "Volume"} de 0 à F.
Cette valeur permet de régler le volume des notes attachées à cet instrument.
\medskip

\begin{itemize}
    \item{0: muet}
    \item{F: à fond !}
\end{itemize}

\paragraph{Enveloppe : "Direction"} \SimpleImage{images/envelope_direction_down}{1.0} ou \SimpleImage{images/envelope_direction_up}{1.0}.
    Indique si le son monte ou descend.

\paragraph{Enveloppe : "Steptime"} de 0 à 7.
Agit sur la longueur du pas de l'onde (en gros).

\paragraph{Enveloppe : "Polystep"} 0 ou 1.
Modifie la facon dont le générateur de bruit fonctionne.
\medskip

\begin{itemize}
    \item{0: bruit simple}
    \item{1: bruit qui grince}
\end{itemize}

\paragraph{Paramètre : "Timed"} 0 ou 1.
\medskip

\begin{itemize}
        \item{0: le son n'est pas temporisé}
        \item{1: le son est temporisé, réglez la durée avec le paramètre "Length"}
    \end{itemize}

\paragraph{Paramètre : "Length"} de 0 à 3F.
Ce paramètre n'est accessible que si "Timed" à été positionné sur 1.
\medskip

\begin{itemize}
    \item{0: le son est "infini"}
    \item{3f: le son est très (très... voire trop) court}
\end{itemize}

\paragraph{Paramètre : "Output"} Choisissez le mode de sortie du son.
\medskip
\begin{itemize}
    \item{L: le son sort à gauche (LEFT)}
    \item{R: le son sort à droite (RIGHT)}
    \item{RL: le son sort à droite et à gauche (LEFT/RIGHT)}
    \item{VIDE: le son ne sort plus (muet)}
\end{itemize}

\paragraph{Simulator} Changez la note inscrite dans cet espace:
            celle-ci sera jouée avec les paramètres de l'instrument que vous êtes en train d'éditer.


\subsubsection{Instrument de type SAMPLE}
\FAT est capable de lire des samples sur les canaux SNA et SNB.
Un sample est un morceau sonore de courte durée inclus dans une collection.
Une série de collection est incluse par défaut dans \FAT :\medskip

\begin{itemize}
    \item {DEV-MESS : quelques samples de test pour le développement}
    \item {LS-MGDRV : des samples Megadrive fournis par Nullsleep (\href{http://little-scale.blogspot.fr/2008/08/sega-mega-drive-sample-pack.html}{Site de Nullsleep})}
    \item {C64DRUMS : drums et snares en provenance du Commodore64}
    \item {YM-DRUMS : drums et snares en provenance de l'Atari ST}
    \item {BATTERY : des kicks, drums et snares d'une batterie "classique"}
    \item {RHYTHMS : quelques ryhtmes complets (mais encore non exploitable car mal samplé)}
    \item {TR-606-7 : sample de cette machine mythique}
\end{itemize} \medskip

Il est assez facile de rajouter votre propre collection de samples. Pour plus d'informations, consultez la section \hyperref[sec:addsamples]{"Manipulation de la ROM pour l'ajout de vos collections de samples"}.

\Image{images/screen_instrument_sample}{0.5}{Type SAMPLE/KIT}

\subsubsection{Paramètres}

\paragraph{Kit collection : "Name"} Le nom de la collection de sample dans laquelle piocher.

\paragraph{Enveloppe : "Volume"} Volume affecté au sample : 50\% (0) ou 100\% (1).

\paragraph{Paramètre : "Loop"} Le sample est t-il joué en boucle ?
                                Si "YES", alors la lecture du sample est répétée.

\paragraph{Paramètre : "Timed"} Définit si le sample sera coupé ou non.

\paragraph{Paramètre : "Length"} Si "Timed" est à 1, cette valeur définit la coupure en pourcentage de la durée totale du sample.

\paragraph{Paramètre : "Offset"} Offset de départ pour la lecture du sample.
                                En pourcentage de la durée totale du sample.

\paragraph{Paramètre : "Output"} Choisissez le mode de sortie du son.
\medskip

\begin{itemize}
    \item{L: le son sort à gauche (LEFT)}
    \item{R: le son sort à droite (RIGHT)}
    \item{RL: le son sort à droite et à gauche (LEFT/RIGHT)}
    \item{VIDE: le son ne sort plus (muet)}
\end{itemize}

\paragraph{Simulator} Changez la note inscrite dans cet espace:
        celle-ci sera jouée avec les paramètres de l'instrument que vous êtes en train d'éditer.


\subsubsection{Instrument de type OSCILLATOR}
\FAT is able to generate the simple sound of an oscillator. These instruments will surely be used one day
For the FM synthesis. In the meantime, they remain usable as such in the SNA and SNB channels.

\ColoredAnnotation {You will soon notice that sinusoidal oscillators do not have a "perfect" sound.
This is due to 2 things. The first one is that the GBA processor does not support floating calculation (numbers with commas).
The calculation of a sinusoidal wave requires a certain precision which can not therefore be obtained natively and easily.
The second thing: it is nevertheless possible to do floating calculation, but in a software way. This kind of calculation is not, however,
still used in \FAT. There is therefore still room for progress on this subject.}

\Image{images/screen_instrument_oscillator}{0.5}{OSCILLATOR type}

\subsubsection{Parameters}

\paragraph{Oscillator : "OSC Shape"} The waveform to generate. \FAT supports 4 types of waves :\medskip

\begin{itemize}
    \item{\SimpleImage{images/osc_shape_sinus}{1.0} sinus}
    \item{\SimpleImage{images/osc_shape_square}{1.0} square}
    \item{\SimpleImage{images/osc_shape_triangle}{1.0} triangle}
    \item{\SimpleImage{images/osc_shape_sawtooth}{1.0} sawtooth}
\end{itemize}\medskip

\paragraph{Oscillator : "Amplitude"} This parameter is not yet available for use.

\paragraph{Enveloppe : "Volume"} Volume assigned to the sample: 50\% (0) or 100\% (1).

\paragraph{Parameter : "Timed"} Sets whether the sample will be cut or not.

\paragraph{Parameter : "Length"} If "Timed" is set to 1, this value sets the percentage cut-off of the total sample time.

\paragraph{Parameter : "Output"} Select the sound output mode.
\medskip

\begin{itemize}
    \item{L: the sound goes out to the left (LEFT)}
    \item{R: the sound goes out to the right (RIGHT)}
    \item{RL: the sound goes out to the left and right (LEFT/RIGHT)}
    \item{VIDE: sound does no go out anymore}
\end{itemize}

