This section will attempt to give you some initial parameter values in order to create "interesting" and get started faster in composing songs.
For each example, I will give you the most complete procedure to follow.

\subsection{Kick "Boom"}

The kick "boom" is programmed on channel 1 exclusively. This is a way to get a kick.
\medskip

\begin{itemize}
\item{in the SONG screen, place the cursor on channel 1 (PU1).}
\item{press L + A to place a new sequence number.}
\item{press SELECT, hold down and select the BLOCKS screen.}
\item{once in the BLOCKS screen, press L + A again to bring up an available block number.}
\item{press SELECT, hold down and select the NOTES screen.}
\item{in the NOTES screen, press A, hold down and select the "C 6" (or "C 7", "C 8") note with the cursor.}
\item{edit the instrument settings. Press SELECT, hold and select the INSTRUMENTS screen.}
\item{the instrument must be of the PULSE type. If this is not the case, change the type by pressing L and going to the left with the cursor.}
\item{boost the volume thoroughly: F.}
\item{the waveduty must be set to the value 2.}
\item{set the sweep value to 41.}
\item{press START !}
\end{itemize}\medskip

Alternatives:
\medskip

\begin{itemize}
\item{by "timing" the instrument, the kick will be drier: "Timed = 1". Then change the value of the "SoundLength".}
\item{try changing the sweep to 31 or 51. A value of 32 or 52 still gives different results.}
\item{the octave of the note is also important! Try to go down to octave 5 and the kick will be less powerful. Conversely, octave 8 makes it possible to slam it harder.}
\end{itemize}\medskip

Screenshot :
\Image{images/kickboom}{1.0}{Instrument - Kickboom}

\subsection{Stereo sample}

By default a sample is mono. However, it is possible to direct sound playback to the right or left with \FAT.
Let's suppose you want to create a stereo sample.
Simply add 2 samples in the ROM (one for the right part and the other for the left) and then to configure 2 instruments
with the OUTPUT parameter adjusted judiciously
  (or use the OR command, which saves you an instrument).
The SNA channel takes care of reading the left one with an OUtput 'Left'.
The SNB takes care of reading the right one with a OUtput 'Right'. The reverse is also possible. It does not matter.
That's it ! You have a stereo sample.
