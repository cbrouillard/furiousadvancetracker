\FAT est capable de lire des samples sur les canaux SNA et SNB.
Un sample est un morceau sonore de courte durée inclus dans une collection.
Une série de collection est incluse par défaut dans \FAT :\medskip

\begin{itemize}
    \item {DEV-MESS : quelques samples de test pour le développement}
    \item {LS-MGDRV : des samples Megadrive fournis par Nullsleep (\href{http://little-scale.blogspot.fr/2008/08/sega-mega-drive-sample-pack.html}{Site de Nullsleep})}
    \item {C64DRUMS : drums et snares en provenance du Commodore64}
    \item {YM-DRUMS : drums et snares en provenance de l'Atari ST}
    \item {BATTERY : des kicks, drums et snares d'une batterie "classique"}
    \item {RHYTHMS : quelques ryhtmes complets (mais encore non exploitable car mal samplé)}
    \item {TR-606-7 : sample de cette machine mythique}
\end{itemize} \medskip

Il est assez facile de rajouter votre propre collection de samples. Pour plus d'informations, consultez la section \hyperref[sec:addsamples]{"Manipulation de la ROM pour l'ajout de vos collections de samples"}.

\Image{images/screen_instrument_sample}{0.5}{Type SAMPLE/KIT}

\subsubsection{Paramètres}

\paragraph{Kit collection : "Name"} Le nom de la collection de sample dans laquelle piocher.

\paragraph{Enveloppe : "Volume"} Volume affecté au sample : 50\% (0) ou 100\% (1).

\paragraph{Paramètre : "Loop"} Le sample est t-il joué en boucle ?
                                Si "YES", alors la lecture du sample est répétée.

\paragraph{Paramètre : "Timed"} Définit si le sample sera coupé ou non.

\paragraph{Paramètre : "Length"} Si "Timed" est à 1, cette valeur définit la coupure en pourcentage de la durée totale du sample.

\paragraph{Paramètre : "Offset"} Offset de départ pour la lecture du sample.
                                En pourcentage de la durée totale du sample.

\paragraph{Paramètre : "Output"} Choisissez le mode de sortie du son.
\medskip

\begin{itemize}
    \item{L: le son sort à gauche (LEFT)}
    \item{R: le son sort à droite (RIGHT)}
    \item{RL: le son sort à droite et à gauche (LEFT/RIGHT)}
    \item{VIDE: le son ne sort plus (muet)}
\end{itemize}

\paragraph{Simulator} Changez la note inscrite dans cet espace:
        celle-ci sera jouée avec les paramètres de l'instrument que vous êtes en train d'éditer.
